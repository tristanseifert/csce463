\documentclass[11pt]{article}

\usepackage[letterpaper, margin=0.74in]{geometry}
\usepackage{lscape}

\usepackage{graphicx}
\usepackage{enumitem}
\usepackage{color,soul}
\usepackage{amsmath}
\usepackage{xfrac}
\usepackage{mathtools}

\usepackage{xcolor}
\definecolor{aggiemaroon}{RGB}{80,0,0}

\usepackage{listings}
\usepackage{caption}
\DeclareCaptionFont{white}{\color{white}}
\DeclareCaptionFormat{listing}{%
  \parbox{\textwidth}{\colorbox{gray}{\parbox{\textwidth}{#1#2#3}}\vskip-4pt}}
\captionsetup[lstlisting]{format=listing,labelfont=white,textfont=white}
\lstset{frame=lrb,xleftmargin=\fboxsep,xrightmargin=-\fboxsep,basicstyle=\tiny\tt}

\usepackage[mathletters,postscript]{ucs}
\usepackage[utf8x]{inputenc}
\usepackage{pmboxdraw}

\usepackage{hyperref}
\hypersetup{
    colorlinks,
    linkcolor=aggiemaroon,
    linkbordercolor=blue,
}
\makeatletter
\Hy@AtBeginDocument{%
  \def\@pdfborder{0 0 1}% Overrides border definition set with colorlinks=true
  \def\@pdfborderstyle{/S/U/W 1}% Overrides border style set with colorlinks=true
                                % Hyperlink border style will be underline of width 1pt
}
\makeatother

\usepackage{tikz}
\usepackage{pgfplots}
\pgfplotsset{compat=1.16}

\usepackage{graphicx}
\graphicspath{{images/}}

\begin{document}
%\newcommand{\answer}[1]{{\bf (Answer: #1)}}
\newcommand{\answer}[1]{\mbox{~}}

{
\large  CSCE 463 \hfill Fall 2020\\
 \begin{center}
   \textbf{Homework 2 Report} \\
   Tristan Seifert \\
    \end{center}
}

%%%%%%%%
\section{Questions}

\begin{enumerate}

%%% (1)
\item {
\texttt{random0.irl} uses compressed offsets pointing into the fixed header. Responses for \texttt{random3.irl} produce a packet that is too small, e.g. smaller than the fixed header. \texttt{random5.irl} has a \textsc{rr} that contains a jump past the end of the packet. Lastly, \texttt{random6.irl} has an infinite jump loop.


\begin{lstlisting}[label=random0-trace,caption=Trace for \textit{random0.irl}]
Lookup  : random0.irl
Query   : random0.irl, type 1, TXID 0x39ce
Server  : 127.0.0.1
********************************
Attempt 1 with 29 bytes... response in 2 ms with 9 bytes
DNS failure: ++ jump into fixed header
\end{lstlisting}


\begin{lstlisting}[label=random3-trace,caption=Trace for \textit{random3.irl}]
Lookup  : random3.irl
Query   : random3.irl, type 1, TXID 0xeb2e
Server  : 127.0.0.1
********************************
Attempt 1 with 29 bytes... response in 2 ms with 9 bytes
DNS failure: ++ response too small
\end{lstlisting}


\begin{lstlisting}[label=random5-trace,caption=Trace for \textit{random5.irl}]
Lookup  : random5.irl
Query   : random5.irl, type 1, TXID 0xc403
Server  : 127.0.0.1
********************************
Attempt 1 with 29 bytes... response in 2 ms with 71 bytes
DNS failure: ++ jump past end of packet
\end{lstlisting}


\begin{lstlisting}[label=random6-trace,caption=Trace for \textit{random6.irl}]
Lookup  : random6.irl
Query   : random6.irl, type 1, TXID 0xc6c8
Server  : 127.0.0.1
********************************
Attempt 1 with 29 bytes... response in 6 ms with 59 bytes
DNS failure: ++ jump loop reconstructing label
\end{lstlisting}
}

%%% (2)
\item {
On \texttt{random1.irl}, the fixed header indicates that there are far more additional records than space is provided for in the packet.

\begin{lstlisting}[label=random1-trace,caption=Trace for \textit{random1.irl}]
Lookup  : random1.irl
Query   : random1.irl, type 1, TXID 0xaef0
Server  : 127.0.0.1
********************************
Attempt 1 with 29 bytes... response in 3 ms with 468 bytes
DNS failure: ++ not enough records
\end{lstlisting}
}

%%% (3)
\item {
Responses for \texttt{random7.irl} are truncated right after the first byte of the two byte jump command. 

\begin{lstlisting}[label=random7-trace,caption=Trace for \textit{random7.irl}]
Lookup  : random7.irl
Query   : random7.irl, type 1, TXID 0x4606
Server  : 127.0.0.1
********************************
Attempt 1 with 29 bytes... response in 3 ms with 468 bytes
DNS failure: ++ truncated jump offset
\end{lstlisting}
}

%%% (4)
\item {
The server chooses randomly whether the response packets from \texttt{random4.irl} have RR value lengths that are too large, jump into the fixed header, or simply truncates the packet in the middle of a name string.

\begin{lstlisting}[label=random4-trace,caption=Traces for \textit{random4.irl}]
Lookup  : random4.irl
Query   : random4.irl, type 1, TXID 0xfcd4
Server  : 127.0.0.1
********************************
Attempt 1 with 29 bytes... response in 3 ms with 468 bytes
DNS failure: ++ unexpected end of packet (name string)


Lookup  : random4.irl
Query   : random4.irl, type 1, TXID 0x2d39
Server  : 127.0.0.1
********************************
Attempt 1 with 29 bytes... response in 3 ms with 468 bytes
DNS failure: ++ invalid jump offset (into fixed header)


Lookup  : random4.irl
Query   : random4.irl, type 1, TXID 0x84cf
Server  : 127.0.0.1
********************************
Attempt 1 with 29 bytes... response in 3 ms with 468 bytes
DNS failure: ++ invalid record length
\end{lstlisting}
}

%%% (5)
\item {
Handling for the \texttt{random8.irl} record implements a specific algorithm, that will overwrite a random part of the packet with repeated instance of the string \texttt{"lol"}. The algorithm uses the system \texttt{rand()} function to get a starting byte offset, as well as to compute how many times the string is repeated: it will always be at least 10 times, but no more than 40.

I determined this by loading the provided \texttt{DNSResponder.exe} binary into Ghidra\footnote{\href{https://ghidra-sre.org}{Ghidra} is a free and open source reverse engineering tool developed by the National Security Agency. It's similar to IDA Pro, which I have some experience in using from years of developing iOS jailbreaks.}, then finding the function responsible for handling the \texttt{random8.irl} record. This was relatively easy, since the compiler generated an easy to recognize jump table for the switch statement that invokes the correct handler, depending on the 0-9/A-B part of the \texttt{randomX.irl} question.

Once the request handler function was found, it was just a matter of following it through to the end to find where the packet is corrupted. Translating the assembly instructions in the executable to pseudocode yields the following:

\begin{lstlisting}[language=c++,label=random8-lol-corruption,caption=Pseudocode of corruption function,keywordstyle=\bfseries\color{green!40!black},commentstyle=\itshape\color{purple!40!black},identifierstyle=\color{blue},stringstyle=\color{orange},]
// generate some random values for packet corruption time
temp = rand();
// starting offset for corruption
corruptionStart = temp % (packetEnd - 0xC)) + 0xC;
// derive the number of lols to write: at least 10, no more than 40
temp = rand();
numLols = temp % 0x1e + 10;

// ensure we don't write past the end of the packet
lolEndByte = corruptionStart + numLols * 3;
if (packetEnd < lolEndByte) {
  corruptionStart = corruptionStart - (lolEndByte - packetEnd);
}

// reset the loop counter
i = 0;

// repeat for number of lols to write
while (i < numLols) {
  // copy three bytes of the lol string
  memcpy((void *)(packet + (i * 3) + corruptionStart), "lol", 3);

  // increment loop counter
  i = i + 1;
}
\end{lstlisting}
}

\end{enumerate}

\end{document}

